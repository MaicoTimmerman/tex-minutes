\documentclass[english]{class/minutes}

\title{Example Minutes}
\date{}

\newcommand{\bob}{\name{Bob Banketbakker}\xspace}
\newcommand{\bor}{\name{Bor de Zwart}\xspace}
\newcommand{\rob}{\name{Rob de Chef}\xspace}
\newcommand{\tim}{\name{Tim Wattenheuvel}\xspace}

\begin{document}

\chairman{Bob Bakker\xspace}
\secretary{Roro de Jongen\xspace}
\starttime{14:25\xspace}
\stoptime{14:30\xspace}
\location{The kitchen\xspace}
\present{\bob, \bor}
\absent{\rob, \tim}

\maketitle
\section{Opening}
\begin{linenumbers}
\timestamp{14:25}{\bob opens the meeting}

%Insert meeting here

% During meetings, people tend to vote. An example of this is added below
\begin{vote}{What will we have for dinner?}
    Pizza & 3 votes\\
    Pasta & 1 vote\\
    Gummy worms with Belgian sprouts & 0 votes\\
\end{vote}
\section{Next meeting}
\name{Raymond Scheepstra} is going to make a year planning. He will decide in what weeks the meetings will take place. Then we can later plan the exact date.
\task{Raymond}{Make a year planning}
\decision{From now one, we will use a year planning}\\
The next meeting will be within a month.\\\\
\bor asks when \tim will do the dishes. \tim says he will do them as soon as possible.
\section{Closing}
\timestamp{14:30}{\bob closes the meeting}

\end{linenumbers}

\tasklist%
\decisionlist%

\end{document}
